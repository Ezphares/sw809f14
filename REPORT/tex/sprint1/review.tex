\section{Review}
\label{sec:sprint1-review}

As described in \autoref{sec:sprint1-tasks}, four tasks had been chosen for the first sprint. The status of these can be seen in \autoref{fig:sprint1-status}. As can be seen, a single task was not completed, and some changes were needed for another. These will be planned for a later sprint. The learning curve of the Android framework will be taken into account when planning the amount of work to be done in later sprints.

\begin{table}[ht!]
	\centering
	\begin{tabular}{|l|p{8cm}|}
		\hline
		\textbf{Task} & \textbf{Status} \\
		\hline
		Phone App & Task done, but learning the Android framework took more time than expected. \\
		\hline
		Route Planning & Task done, but will need a better user interface. \\
		\hline
		Ranking & Task done. \\
		\hline
		\ac{GPS} & Task not done. \\
	\hline
	\end{tabular}
	\caption{Sprint 1 Task Status}
	\label{fig:sprint1-status}
\end{table}

All in all, we considered the first sprint a success and a learning experience, which gave us a better idea of how to approach Android development, as well as determining the skill of players.
