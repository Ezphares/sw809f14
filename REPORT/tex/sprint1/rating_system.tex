\section{Rating System}
The purpose of a rating system is to keep track of the relative skill levels of players.
A rating system in its most basic form accomplishes this by representing the skill level of a player as a number of points, which is decreased when the player loses a match, and increased when the player wins a match.
The exact number of points by which the rating is increased or decreased, and how often this is done etc. depends on the particular rating system.

In order to automatically match users against each other we need a way to determine their relative skill levels.
This is also needed in order to provide users with feedback on their personal progress, for example by creating leaderboards or by providing users with statistics about their increase or decrease in rating.

A number of rating systems already exist, which are in use in different types of games such as sports, board games and video games.
In this section some of the most well known rating systems will be analyzed.
The rating system used in this project will also be described in terms of implementation and testing.

\subsection{Elo}
\label{sec:elo}
The Elo rating system is a method for determining the relative skill levels of players in two-player games.
It was created in the early 1960's by Arpad Elo, and modified versions of it are still in use today by the World Chess Federation (FIDE) and the United States Chess Federation (USCF), among others.

The difference in ratings between two players is used to predict the outcome of a match.
Suppose that player A has rating $R_A$ and player B has rating $R_B$, then the expected score for player A is calculated as shown in Equation \ref{eq:exp_score}.
The expected score for player B is calculated as $E_B = 1 - E_A$.

\begin{equation} \label{eq:exp_score}
E_A = \frac{1}{1 + 10^\frac{R_B - R_A}{400}}
\end{equation}

The updated rating $R'_A$ for player A is given by Equation \ref{eq:update}, where $S_A$ is the actual score (1 = win, 0.5 = draw, 0 = loss), $E_A$ is the expected score for player A and $K$ is the K-factor.
The K-factor governs the volatility of changes to ratings.
Often a high K-factor is used for new players where there is greater uncertainty about their true rating, and a low K-factor is used for experienced players.

\begin{equation} \label{eq:update}
R'_A = R_A + K \left( S_A - E_A \right)
\end{equation}

For example, suppose player A and player B both have ratings 1700 and player A defeats player B.
The expected scores for both players are given by Equation \ref{eq:exp_score_ex}, and their updated ratings are given by Equations \ref{eq:update_ex_a} and \ref{eq:update_ex_b}, assuming a K-factor of 32.

\begin{equation} \label{eq:exp_score_ex}
E_A = E_B = \frac{1}{1 + 10^\frac{1700 - 1700}{400}} = 0.5
\end{equation}

\begin{equation} \label{eq:update_ex_a}
R'_A = 1700 + 32 \left( 1 - 0.5 \right) = 1716
\end{equation}

\begin{equation} \label{eq:update_ex_b}
R'_B = 1700 + 32 \left( 0 - 0.5 \right) = 1684
\end{equation}

\subsection{Glicko and Glicko2}
\label{sec:glicko}
The Glicko rating system was created in 1995 by Mark E. Glickman to address a deficiency that he identified in the Elo rating system \cite{glicko}.

Glickman uses the same example as the one given in Section \ref{sec:elo} to illustrate the problem.
Suppose that player A has not competed for a long period of time, and that player B competes frequently.
This means that the rating of player A is less accurate than the rating of player B.
He argues that:

\begin{itemize}
	\item{The rating of player A should increase by more than 16 points because it is already not accurate, and defeating a player with a fairly accurate rating of 1700 is a reasonable indication that the true rating of player A is higher than 1700.}
	\item{The rating of player B should decrease by less than 16 points because the rating of player B is already fairly accurate, and because the rating of player A is not accurate very little can be inferred about the skill level of player B.}
\end{itemize}

To achieve this a \emph{ratings deviation} (RD) is introduced, which is a measure of the uncertainty of a rating.
A high RD equates to a high uncertainty, and a low RD equates to a low uncertainty.
The RD of a player is always decreased when the player competes, and increases as time passes.

The Glicko rating system treats games that occurred within the same \emph{rating period} as if they occurred simultaneously.
The length of a rating period ranges from several months to a game-by-game basis depending on the particular application.

Before determining the post-period rating and RD of a player, the pre-period RD of the player must first be updated.
This is shown in Equation \ref{eq:pre_rd}, where $t$ is the number of rating periods since the player last competed, and $c$ is a constant that controls the volatility of increase in uncertainty over time.

\begin{equation} \label{eq:pre_rd}
RD = \min \left( \sqrt{RD_{old}^2 + c^2t}, 350 \right)
\end{equation}

The post-period rating of a player is given by Equation \ref{eq:post_rating} and the post-period RD is given by Equation \ref{eq:post_rd}.

\begin{equation} \label{eq:post_rating}
r' = r + \frac{q}{\frac{1}{RD^2} + \frac{1}{d^2}} \sum \limits_{j=1}^m g \left( RD_j \right) \left( s_j - E \left( s | r, r_j, RD_j \right) \right)
\end{equation}

\begin{equation} \label{eq:post_rd}
RD' = \sqrt{\left( \frac{1}{RD^2} + \frac{1}{d^2} \right)^{-1}}
\end{equation}

The value of q is $\frac{ln \left( 10 \right)}{400} = 0.0057565$ and the other variables are given by Equations \ref{eq:glicko_g}, \ref{eq:glicko_e} and \ref{eq:glicko_d2}.

\begin{equation} \label{eq:glicko_g}
g \left( RD \right) = \frac{1}{\sqrt{1 + \frac{3 q^2 RD^2}{\pi^2}}}
\end{equation}

\begin{equation} \label{eq:glicko_e}
E \left( s | r, r_j, RD_j \right) = \frac{1}{1 + 10^{\frac{-g \left( RD_j \right) \left( r - r_j \right)}{400}}}
\end{equation}

\begin{equation} \label{eq:glicko_d2}
d^2 = \left( q^2 \sum \limits_{j=1}^m g \left( RD_j \right)^2 E \left( s | r, r_j, RD_j \right) \left( 1 - E \left( s | r, r_j, RD_j \right) \right) \right)^{-1}
\end{equation}

In the example given in Section \ref{sec:elo} the rating of player A was 1716 and the rating of player B was 1684.
Using the Glicko rating system instead, and assuming that player A has an RD of 200 and player B has an RD of 50, the post-period rating of player A is 1785 and the post-period rating of player B is 1694.
This shows that the Glicko rating system has the desired effect of increasing the rating of player A by a large amount while only decreasing the rating of player B by a small amount.

Glickman has also proposed an extension to the original Glicko rating system called the Glicko 2 rating system \cite{glicko2}.
In the Glicko 2 rating system every player still has a rating and an RD, but also a rating volatility $\sigma$.
The rating volatility for a player is high when the player performs erratically and low when the player performs at a consistent level.
The Glicko 2 rating system is most suitable for games with extremely unpredictable outcomes.

\subsection{Implementation}
For our application we use the original Glicko rating system, in order to benefit from the mentioned improvements it provides over the Elo rating system.
The Glicko rating system is preferred over the Glicko 2 rating system as we do not expect the games to have extremely unpredictable outcomes.

The default rating for new players is set to 1500 with an RD of 350, as these are the values recommended by Glickman.
The minimum RD is set to 30 to ensure that the rating of a player can always change appreciably.
The rating period is set to one day, and matches will be scored on a game-by-game basis as soon as they are over.

The value of the constant $c$ is set to 18.1.
This was determined by using the method proposed by Glickman \cite{glicko}, which is to determine how many ratings periods it would take for the rating of a typical player to become as uncertain as the rating of an unranked player.
These numbers cannot be determined in an exact manner, so we make an estimation that this would take approximately 365 rating periods (one year), and that the RD of a typical player is approximately 50.
These numbers were then used to solve Equation \ref{eq:pre_rd} for $c$.

The Glicko rating system is implemented as a single Python class.
The implementation is straightforward as it only contains the six equations shown in Section \ref{sec:glicko}.

\subsection{Testing}
Every method in the \texttt{Glicko} class has been unit tested.
The \texttt{Glicko} class is very suitable for unit testing because it has no external dependencies or side effects.
A total of 7 tests cases and 10 assertions were written, giving a code coverage of 97 \%.
