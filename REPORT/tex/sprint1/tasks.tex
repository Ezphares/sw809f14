\section{Tasks}
\label{sec:sprint1-tasks}

For the first sprint we decided to start with some of the fundamental functionality that would be needed to create an outline of the prototype described in \autoref{sec:problem-exp}. Our choice was the following functionality:

\begin{itemize}
	\item{A skeleton for the mobile application.}
	\item{Route planning, including saving and retrieving routes.}
	\item{A system for ranking users.}
	\item{Access to \ac{GPS} tracking data on the mobile device.}
\end{itemize}

Before these tasks could be started a few group decisions had to be made. The biggest of these decisions was for which mobile platform the application should be developed. We quickly decided to use Android, as the entire group had experience with using that platform, and we had access to several Android devices which would make testing easier.

Another decision was reached, that the route planning system should be accessed on a web page, and routes stored in a central database. This decision was made for two reasons: First, we decided it would be easier to create and edit routes on a map using a mouse rather than a touch interface. Second, we wanted to store the routes in a central database anyway, as we want users to have access to all their routes from any compatible device they own.
