\section{Review}
\label{sec:sprint3-review}

The tasks chosen for this sprint were presented in \autoref{sec:sprint3-tasks}, and the result of the sprint can be seen in \autoref{fig:sprint3-status}. This sprint all tasks were marked as done, and referencing \autoref{sec:sprint1-review} and \autoref{sec:sprint2-review}, this was the first sprint where this was the case.

\begin{figure}[ht!]
 \caption{Sprint 3 Task Status}
 \label{fig:sprint3-status}
 \begin{tabular}{|l|l|}
  \hline
  \textbf{Task} & \textbf{Status} \\
  \hline
  Application as Match Client & Done \\
  \hline
  Match Communication Protocol & Done \\
  \hline
  Server Side Route Tracking & Done \\
  \hline
 \end{tabular}
\end{figure}

This sprint was considered a success. All tasks were completed, even the recurring \ac{GPS} tracking task. This meant that at this point the fundamental features were done, and going forward, we could start focusing on some of the things that makes our system different from existing solutions.
