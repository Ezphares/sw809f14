\section{Reflection}
Overall the project process went well, however there were some bumps in the road. For example the project planning was done at the start of the semester, which was very good, but as the semester progressed some tasks were not complete, deadlines were missed and this cause the planning to fall apart. In the future we plan to leave more room for such problems in the planning to avoid being stressed out at the end of the semester.

The development method, \ac{XP}, was underutilized during development. The group started out adhering to the principles, however as time went on and the deadline approached the method was no longer followed. This should be done better next time but it should be easier to stick with it with less stress so if the planning is better the utilization of the development method should also improve.

During development the group spent some time implementing features that were not utilized in the final proof of concept, such as friend list. However the implementations are sound and could be used in future development so they are not wasted per se. However the time would probably have been better spent refining the utilized features. In the future we plan to spend more time reflecting on the functionality we implement and if it is usable in the project.

The use of Essence helped identify and strengthen the innovative aspects of the project idea and we truly think that this is a novel addition to the existing solutions on the market.

The project currently has a very modular structure with easy refactoring or replacement of individual modules. This is a good thing as it allows for gradual improvement of the system without having to re-implement everything. The chosen client-server architecture works well for this type of project.

The smartphone application is very lightweight as currently all heavy computations are done server-side. It does expend a lot of battery at the moment however some of this can be remedied through changing updating frequencies on the \ac{GPS}. 
