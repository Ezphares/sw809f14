\section{Essence}
Essence is a methodology developed by Ivan Aaen to facilitate innovation in software development projects. It aims to synthesize the values from the traditional and agile paradigm and create a new view of software innovation. The core of Essence is the use of concepts called Values, Views, Roles and Visions. Below is a very brief description of the concepts the group chose to utilize in the exploration of the problem domain. The following is based on "Essence - Pragmatic Software Innovation" \cite{essence}.

\subsection{Values} 
Essence describes four difference valus 
\begin{itemize}
\item \textbf{Reflection} refers to the concept of \textit{Reflection over requirements}, i.e. the value that understanding the needs of the customer is more important than listing the specifications for the end product. This is new compared to the agile and traditional paradigm that both expect the customer to know and formulate their needs.  
\item Affordance is a value that encapsulates \textit{Affordance over solution} which means that discovering new options is more important than delivering raw results. In the traditional and agile paradigm the main focus is on delivering a final solution that is as close a match to the customer's definition as possible. 
\item The Vision value encompasses \textit{Vision over assignment}, i.e. that it is more important to know where you want to go than to know how you get there. Traditional and agile methods manage their projects through assignments to be done where Essence focuses on shaping and exploring the vision of the project in order to realize which direction the project should be taken in.
\item Facilitaion is about the concept of \textit{Facilitation over structuration}. This value means that moving in the right direction is more important that following a set routine. It sees the process of software development as a teamwork facilitator and should provide support for reflection within the team to help discover new processes and possibly building new paradigms.
\end{itemize}

\subsection{Views} 
In Essence there are four views

\begin{itemize}
\item Paradigm is a view that deals with how the challenge of the project could be solved with a user interface. It should reflect the team's idea of use context and of the problem domain in which they work. It is related to the Reflection value and the Child role.
\item Product deals with implementation with a focus on architecture and how to build functionalities. It represents the solution domain and is related to the Affordance value and the Responder role.
\item The Project view is a view that deals with planning, project status and how to prioritize the project. It is related to the vision value and the Challenger role. 
\item Process deals with creativity facilitation, understanding of the challenge and evaluation and maturation of new ideas. It is connected to the Facilitation value and the Anchor role. 
\end{itemize} 

\subsection{Visions} 
Visions are representations of the ideas that shape the direction of the team effort. These cam be expressed in four different ways, with different characteristics: simple or complex, concrete or abstract.
\begin{itemize}
\item An Icon is a concrete and simple symbol that illustrates an important quality of an idea. It should serve as inspiration for the team and should prompt the team to ask themselves what they could do to create the same experience of quality in their solution.
\item A Prototype is a concrete and complex representation of a vision that expresses a set of ideas. They are the unfinished software used to communicate ideas and designs and to allow the team to explore various interface options. 
\item A Metaphor is an abstract, simple symbol used to convey an important aspect of an idea. Due to the figurative nature of a metaphor they are open to interpretation, which should facilitate creativity and are used to open up discussion about the design of a product. 
\item A Proposition is an abstract, complex vision that describes a core part of the project. It describes characteristic features of the project and expresses the purpose of the project. A proposition should be well-founded so it can be defended if challenged.
\end{itemize} 
