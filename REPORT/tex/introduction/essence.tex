\section{Essence}
Essence is a methodology to facilitate innovation in software development projects. It aims to synthesize the values from the traditional and agile para\-digms and create a new view of software innovation. The core of Essence is the use of concepts called Values, Views, Roles and Visions. Below is a brief description of the concepts the group chose to utilize in the exploration of the problem domain. The following is based on ``Essence - Pragmatic Software Innovation''~\cite{essence}.

\subsection{Values} 
Essence describes four different values: 
\begin{itemize}
\item \textbf{Reflection} refers to the concept of \textit{Reflection over requirements}, i.e. the value that understanding the needs of the customer is more important than listing the specifications for the end product. This is new compared to the agile and traditional paradigms because both expect the customer to know and formulate their needs.  
\item \textbf{Affordance} is a value that encapsulates \textit{Affordance over solution} which means that discovering new options is more important than delivering raw results. In the traditional and agile paradigms the main focus is on delivering a final solution that is as close a match to the customer's definition as possible. 
\item The \textbf{Vision} value encompasses \textit{Vision over assignment}, i.e. that it is more important to know where you want to go than to know how you get there. Traditional and agile methods manage their projects through assignments to be done where Essence focuses on shaping and exploring the vision of the project in order to realize which direction the project should take.
\item \textbf{Facilitation} is about the concept of \textit{Facilitation over structuration}. This value means that moving in the right direction is more important than following a set routine. It encompasses evaluation of work processes and how they can support the team. The focus has been shifted from viewing processes and methods as rigid stabilizers of the project to viewing them as tools to support improvisation and flexibility.
\end{itemize}

\subsection{Views} 
In Essence there are four views:

\begin{itemize}
\item \textbf{Paradigm} is a view that deals with the team's understanding of the use context and of the problem domain of the challenge. It is related to the Reflection value.
\item \textbf{Product} deals with the technology domain. It focuses on architecture and how to implement functionality. It represents the solution domain and is related to the Affordance value.
\item The \textbf{Project} view is a view that deals with project management such as planning, project status and how to prioritize the functionality of the project. It is related to the Vision value. 
\item \textbf{Process} deals with creativity facilitation, understanding of the challenge and evaluation and maturation of new ideas. It is connected to the Facilitation value. 
\end{itemize} 

\subsection{Visions} 
Visions are representations of the ideas that shape the direction of the team effort. These can be expressed in four different ways, with different characteristics: simple or complex, concrete or abstract.
\begin{itemize}
\item An \textbf{Icon} is a concrete and simple symbol that illustrates an important quality of an idea. It should serve as inspiration for the team and should prompt the team to ask themselves what they could do to create the same experience of quality in their solution.
\item A \textbf{Prototype} is a concrete and complex representation of a vision that expresses a set of ideas. It is the unfinished software that is used to communicate ideas and designs and to allow the team to explore various interface options. 
\item A \textbf{Metaphor} is an abstract, simple symbol used to convey an important aspect of an idea. Due to the figurative nature of a metaphor they are open to interpretation, which should facilitate creativity and is used to open up discussion about the design of a product. 
\item A \textbf{Proposition} is an abstract, complex vision that describes a core part of the project. It describes characteristic features of the project and expresses the purpose. A proposition should be well-founded such that it can be defended if challenged.
\end{itemize} 
