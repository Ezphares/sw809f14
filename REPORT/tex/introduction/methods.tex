\section{Development Method}
When developing a software system there are several different methods for doing so. These can be roughly split into two methodologies, traditional and agile, and a few which might fall between or outside this definition. Traditional development methods, such as the waterfall method, focus on taking one activity at a time, and completing and documenting this activity before moving on to the next, typically in a sequence of analysis, design, implementation, testing. Agile methods on the other hand usually work in iterations, doing some of every activity in each iteration and basing the work in each iteration on what was learned in the last one.

\subsection{Agile Development}
We decided to use an agile development method. This has several advantages, one of which is the ability to explore the problem domain from iteration to iteration, and given that the group has little experience with developing phone applications, this was seen as very valuable. Additionally, the project has to be completed in a single semester, there are likely features that would be nice to have but will end up being outside the scope of a relatively short time frame, which means time will not be spent on detailed analysis and design of features that will not be implemented. Finally we felt that the values described in the agile manifesto will lead to a better product. For reference, the agile manifesto is quoted here:

\begin{quotation}
``Individuals and interactions over processes and tools

Working software over comprehensive documentation

Customer collaboration over contract negotiation

Responding to change over following a plan''~\citep{shitmanifesto}
\end{quotation}

\subsection{Extreme Programming}
There are several development methods within the agile methodology. We decided to use a method called \ac{XP}~\citep{beck04}, a very approachable method, which was desirable because the members of the group had little overlap in their experience with different agile methods. Another point that made \ac{XP} attractive was its relatively few development artifacts, and high focus on producing quality code.

\subsection{Development Method and the Report}
The structure of the rest of this report will be heavily influenced by the choice of an agile development method. Choices, design, and implementation will be described on per-sprint basis, with each sprint having a chapter on its own. Each of these chapters will begin with the choices made for that sprint, and end with a status review of what was done. Everything in between is a description of the development process of specified features or systems done in the sprint.
