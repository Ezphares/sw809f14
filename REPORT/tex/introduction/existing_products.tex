\section{Existing Products}
Several products utilizing gamification of exercise already exist, with varying focus. In this section the popular applications "RunKeeper" and "Zombies, Run!" are examined. 

\subsection{RunKeeper}
The application RunKeeper combines a mobile client with an online website and keeps track of workouts, personal records, goals and personal records. It utilizes gamification to good effect by implementing rankings between friends based on the number of workouts done that month, a "like" functionality for workout updates, a variety of achievements for breaking personal records (altitude climb, speed, distance, number of workouts etc) and live tracking of one's activity so friends and family can follow one's workout in real time (only available in the premium version). The basic app and an online profile is free and a premium version is available for a fee, with added tracking/logging functionality, personalized workouts and, as mentioned, live broadcasting of workouts. RunKeeper also offers workout plans such as a \ac{C25K} plan that, if followed, should enable the user to run a 5 kilometre race. The site utilizes an activity feed similar to Facebook, with updates from friends about workouts and the option of commenting and liking the update. 
\vspace{10pt}

The RunKeeper app has most of the functionality of the website which means you can exclusively use either one without the other. While this is practical for some users it means that a user with an older smartphone can have problems running the app. The gamification of the activities is quite well done. The focus of the app is mostly on a social, friend-based application which motivates users through friendly competition and reminders to work out after periods of inactivity. There is no option to race against another person and the achievements are purely based on personal records. The ranking system only ranks the user among their friends. While competing against friends is attractive to many people more competitively inclined users, or very skilled athletes might not have any use of this, save for motivation from friends and showing off physical prowess.
\vspace{10pt}

Features from RunKeeper to consider for the semester project are:
\begin{itemize}
\item Achievements
\item Scoreboard
\item Personal Goals
\item Workout Reminders
\item Extensive logging
\end{itemize}

\subsection{Zombies, Run!}
This application is a gamified application targeted towards runners. The application takes the form of an audio drama with the user as the main character. During a workout the user listens to their own playlist of music interleaved with sound clips telling the story of the township of Abel after a zombie apocalypse and the adventures of the main character known as "Runner 5". Interval training, called "zombie chases", can be switched on or off before a workout starts. While the user runs random items are collected which can be used in a small in-app game which revolves around the design and expansion of the Abel township from a tiny settlement to a larger township. The app can be connected to e.g. RunKeeper to allow the user to track their runs on RunKeeper while listening to the story of Runner 5 during their runs. \ac{GPS} tracking is available and is, among other things, used to determine if the user speeds up during the zombie chases. An accelerometer tracking option is also available, counting the number of steps a user takes during a run.
\vspace{10pt}

Zombies, Run! relies on integration with other trackers to provide social aspects. There is no option to run with friends. The app uses achievements and an in-app mini-game to motivate the user. The focus is on one person instead of being a social app but with the option to include social aspects through the aforementioned integration. There is minimal tracking: number of completed missions (runs), distance, time and items collected.
\vspace{10pt}

Features from Zombies, Run! to consider for the project:
\begin{itemize}
\item Audio motivation/feedback
\item Focus on personal improvement
\item No innate social aspect
\item New experiences during every run via the audio logs
\item Personalization (listen to your own music during a run)
\end{itemize}