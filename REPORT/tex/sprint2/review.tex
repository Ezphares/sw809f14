\section{Review}
\label{sec:sprint2-review}

In \autoref{sec:sprint2-tasks} we described the tasks chosen for this sprint. The status for of these can be seen in \autoref{fig:sprint2-status}. As shown the \ac{GPS} tracking was left incomplete again. This is unfortunate, but as mentioned in \autoref{sec:sprint2-app}, a lot of unexpected time was spent porting the application project to a different \ac{IDE} and build system. Still, had communication been better, some work might have been diverted less important task, such as the \ac{UI} overhaul of the web page to the \ac{GPS}, which, after all, is important functionality for the system to work.

\begin{figure}[ht!]
 \caption{Sprint 2 Task Status}
 \label{fig:sprint2-status}
 \begin{tabular}{|l l|l|}
  \hline
  \textbf{Task} && \textbf{Status} \\
  \hline
  Application & Map Integration & Done \\
   & Route Display & Done \\
   & \ac{GPS} Tracking & Not Done \\
  \hline
  Matchmaking System && Done \\
  \hline
  Web Page & Elevation Profile & Done \\
   & Friend List & Done \\
  \hline
 \end{tabular}
\end{figure}

We could not call this sprint a complete success, because of the still missing \ac{GPS} tracking, which might limit the possibilities for choosing tasks for later sprints. This was emphasized by the fact that this is a fundamental feature of the system, that has been unsuccessfully planned for two sprints. However, progress was made on several other aspects of the project, so the sprint was still considered a partial success.
