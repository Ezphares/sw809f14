\section{Tasks}
\label{sec:sprint2-tasks}

As described in \autoref{sec:sprint1-review} the first sprint resulted in a good baseline system to build upon. This made it easier to pick more well defined tasks for this sprint, as the group members quickly found some features which they felt was the most needed to move closer to the goal. The chosen tasks can be seen below:

\begin{itemize}
	\item Application features:
	\begin{itemize}
		\item Map integration.
		\item Route display.
		\item \ac{GPS} tracking.
	\end{itemize}
	\item Matchmaking system.
	\item Website features:
	\begin{itemize}
		\item Elevation profiles.
		\item Social features - friend list.
	\end{itemize}
\end{itemize}

The choices made for features in the application seemed obvious to all of the group members. We wanted the map view where the chosen route is visible to be the default view for a user during a race, and the \ac{GPS} integration was a feature not completed in the first sprint.

For the matchmaking system, two different choices were discussed: Either a standalone matchmaking system should be created and communication with the application done later or a communications protocol should be created first, pushing the actual matchmaking system to later sprint. We decided to go with the matchmaking system first, because we decided that knowing the exact data needed by the system would allow us to better create the communications protocol later.

Finally some more features were desired on the website. Elevation profiles are very important for users planning a route, and could be used for competitive features such as determining a ``difficulty'' for a route. The friend list is a preparation for the functionality of challenging a friend to a race directly.
